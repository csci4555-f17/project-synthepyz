\documentclass{article}
\usepackage{todonotes}
\usepackage{tikz}
\usepackage{listings}

% Insert liner notes using these commands
\newcommand{\nlsays}[1]{\todo[color=orange!40]{NL: #1}}
\newcommand{\sdsays}[1]{\todo[color=blue!40]{SD: #1}}
\newcommand{\wtsays}[1]{\todo[color=green!40]{WT: #1}}

\title{Matching Reference Behavior \\ by Simple Enumerative Synthesis}

\author{
  Souradeep Dutta \and
  Nicholas V. Lewchenko \and
  William Temple
}

\begin{document}

\maketitle

\nlsays{Just my first idea for the title; suggest your own}

\nlsays{Authors are alphabetical; we can rearrange by significance or
  whatever if desired}

\section{Introduction}
\label{sec:introduction}

\begin{figure}
  \centering
  \begin{tikzpicture}[node distance=5cm]
  \node (python) {{\Large \texttt{x ** y}}};
  \node (a) [above of=python,yshift=-4.2cm] {(a)};
  \node (theory) [right of=python] {{\LARGE $x^y$}};
  \node (b) [above of=theory,yshift=-4.2cm] {(b)};
  \node (x86) [right of=theory] {\lstinputlisting[basicstyle=\small]{exp.txt}};
  \node (c) [above of=x86,xshift=-2.3cm,yshift=-4cm] {(c)};
  \node [right of=python,xshift=-2.5cm] {{\LARGE$\rightarrow$}};
  \node [right of=theory,xshift=-3cm] {{\LARGE$\rightarrow$}};
\end{tikzpicture}

%%% Local Variables:
%%% mode: latex
%%% TeX-master: "paper"
%%% End:

  \caption{An exponentiation node in Python (a) translated into the
    arithmetic theory definition (b) and then x86 assembly
    implementation (c).}
  \label{fig:exp}
\end{figure}

Writing a compiler is difficult, in part because it requires a
programmer to implement the core functionality of their source
language in one or more assembly languages.
%
An assembly language is optimized for simple evaluation by a computer
(simple enough that \emph{hardware} can do it).
%
The consequence of this design requirement, illustrated in
Figure~\ref{fig:exp}, is that non-trivial assembly programs grow
large, and writing or reasoning about them is tedious and error-prone
for a human.
%
The difference in complexity between the fully-specifying formal
definition of an operation and its assembly implementation is
dramatic, suggesting that computer assistance in this implementation
process would significantly reduce the necessary programming effort of
a compiler programmer.

\paragraph{Background}
\label{sec:background}

The field of \emph{program synthesis} aims to automatically perform
this specification-to-implementation process \ldots

\paragraph{Contributions}
\label{sec:contributions}

We have defined and implemented an enumerative synthesis procedure
called \textsc{SomeName} \nlsays{Will should give the tool a name.} for
operation implementations in a compiler for which a reference
implementation is available.  Our technique is source-language
agnostic, using the preexisting reference implementation to produce a
fixed set of test cases against which x86 programs are sought.  This
reference-guided partial verification reduces the complexity of the
technique.

\section{Our Approach}
\label{sec:approach}

\ldots
\nlsays{Lay out the ideas and then whatever formal theory we have}

\subsection{Implementation}
\label{sec:implementation}

\ldots
\nlsays{Discuss implementation details and technology}

\section{Evaluation}
\label{sec:evaluation}

\ldots

\section{Related Work}
\label{sec:related}

\ldots
\nlsays{Should be very short for us; discuss differences to previously
  demonstrated enumerative synthesis}

\section{Conclusion}
\label{sec:conclusion}

\ldots

\end{document}

